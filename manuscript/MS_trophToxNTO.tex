\documentclass[11pt]{article}

\usepackage[utf8]{inputenc}
\usepackage[T1]{fontenc}
%\RequirePackage{lmodern}	% font set: Latin Modern
\RequirePackage{charter}	% font set: Charter

\usepackage[french,english]{babel}

% Package ADDED
\usepackage{xcolor,graphics,graphicx}

% For citations
\usepackage{csquotes}

% For line spacing
\usepackage{setspace}
\linespread{1.3}

% To rotate a table or a figure
\usepackage{lscape}
% MATHS
%% The amssymb package provides various useful mathematical symbols
\usepackage{amssymb}
\usepackage{amsmath}
\usepackage{empheq}

% TAILLE de la page
\usepackage{geometry} % define page margin
\geometry{top=20mm,left=30mm,right=30mm,bottom=20mm} % margin from top, left, right and bottom

% COMPTEUR de ligne
%% The lineno packages adds line numbers. Start line numbering with
%% \begin{linenumbers}, end it with \end{linenumbers}. Or switch it on
%% for the whole article with \linenumbers after \end{frontmatter}.
\usepackage{caption}
\usepackage{lineno}
\linenumbers

% BIBLIOGRAPHIE
%\usepackage[numbers]{natbib}
\usepackage[authoryear]{natbib}
\usepackage{hyperref}

\begin{document}

\begin{center}
	\Large\textbf{
	Trophic transfer of anticoagulant rodenticides while managing rodent pests: the fine line between predator-prey regulation and pesticide-pest regulation
}\par
\end{center}

\vspace{.5cm}

\begin{center}
	\large\textbf{
		Virgile Baudrot$^{1,2}$ 
		Javier Fernandez-de-Simon$^1,3$
		Michael Coeurdassier$^1$,
		Geoffroy Couval$^1,4$,
		Patrick Giraudoux$^1$,
		Xavier Lambin$^5,6$		
	}\par	
\end{center}

\begin{center}
	$^1$ Université Bourgogne Franche-Comté - UMR CNRS 6249 Laboratoire Chrono-environnement, 25030 Besançon, France\\
	$^2$ BioSP, INRA, 84000 Avignon, France\\
	$^3$ IREC. Instituto de Investigación en Recursos Cinegéticos\\
	$^4$ FREDON Franche-Comté, Espace Valentin Est, 12, Rue de Franche-Comté - Bât E, 25480 Ecole-Valentin, France\\
	$^5$ School of Biological Sciences, University of Aberdeen, Zoology Building, Tillydrone Avenue, Aberdeen AB24 2TZ, Scotland, UK\\
	$^*$ Corresponding authors: virgile.baudrot@posteo.net or x.lambin@abdn.ac.uk
\end{center}

\paragraph{Running Title} Bufferscapes to preserve Non-Target Organism / Ecosystem services

\vspace{.5cm}

\begin{abstract} ~
	\begin{enumerate}
		\item Understanding pesticide impacts on populations of target/non-target species and communities is a challenge to applied ecology. When predators that otherwise regulate pest densities ingest prey contaminated with pesticides, this can suppress predator populations as contaminated prey act as a super-predator, with pesticides controlling both pest and predators. It is, however, unknown how species relationships and protocols of rodenticide treatments, i.e. farmer functional responses, interact to affect pest regulation.
		\item Here we used linked differential equations to model a heuristic non-spatialized system including montane water voles, specialist vole predators (stoats, weasels), and a generalist predator (red fox) which consume voles, mustelids and other prey. By considering anticoagulant rodenticide toxicokinetic and toxicodynamic equations, we explored the impact of 5 farmer functional responses (defined by both rodenticide quantity and threshold vole density above which rodenticide spreading is prohibited) on predator-prey interactions, rodenticide transfer across the trophic chain and population effects.
		\item Spreading low quantities of rodenticide while maintaining sufficient voles as prey resources led to less rodenticide being applied and extended periods without rodenticide in the environment, benefitting predators while avoiding episodes with high vole density. This may meet farm production interests while minimizing the impact on small mustelid and fox populations.
		Spreading rodenticide at low vole densities suppressed mustelid and fox populations, leading to vole population dynamics being entirely regulated by rodenticide use. This vole-eradication treatment regime inhibited predator ecosystem services and promoted pesticide dependence.
		\item Farmer functional response with high or intermediate densities thresholds triggering treatment produced vole population fluctuations caused by rodenticide that could be followed by periods during which mustelids regulated vole populations. These alternative phases of mustelids and farmer regulation highlight the benefit of intraguild relationship where mustelids may rescue foxes from poisoning.
	\end{enumerate}

\paragraph{Synthesis and applications} Different farmer functional responses lead to a rich variety of population dynamics in predator-prey systems. That such a pesticide-tri-trophic system may cause a variety of population dynamics responses to pesticide use in agro-ecosystems is a novel insight. Our model reveals the need for maintaining refuges with sufficient non-poisoned voles for specialist mustelids, to conserve predator community, given the super-predator role of rodenticides. We suggest that long periods without pesticide treatment are essential to maintain predator populations, and that practices of pesticides use that attempt to permanently eradicate a pest over a large scale are counterproductive. 

\end{abstract}

\paragraph{Keywords} Biodiversity conservation; bromadiolone transfer; cyclic fluctuations; pesticides impact; secondary poisoning; super-predation; cascade effects

\section{Introduction}

Since the “green revolution” following the 1950s, pesticides use has increased to control pests damaging properties, public health or crops \citep{Tilman2002}. Pesticides usage is varyingly triggered by the perception/estimation of pest densities. Natural enemies (e.g., predators, parasites, competitors) also reduce pest densities and hence may preclude the need for using pesticides \citep{Michalko2017}. Natural enemies are, however, also affected by pesticides, either by direct exposure, through ingestion of contaminated prey \citep{Berny2007} or indirectly by cascade effects of resource depletion \citep{Halstead2014}.  Thus, under some regimes of pesticides use, pest populations only become regulated by pesticides once predators have collapsed. To preserve ecosystem health and the services of predators through regulation of pest densities, we need to assess the feasibility and benefit of pesticide treatment regimes in their ability to control pest species with minimal damage on predators \citep{Halstead2014}. It is however empirically challenging to assess the overall impact of pesticide treatments on the dynamics of species linked by trophic interactions. In this context, process-based models describing simplified scenarios are powerful tools to reveal hidden patterns by disentangling processes emerging from pesticide impacts on predator-prey system (e.g., \citet{Baudrot2018}).

Voles and other grassland rodent species undergo multi-annual population cycles (e.g.,  \cite{Krebs2013}. At their peaks, they may attain extremely high densities, causing substantial damage to grass/cultivated crops and forestry and conflicts with humans \citep{Delattre2009}. Farmers worldwide expand financial resources to purchase and spread anticoagulant rodenticides (hereafter AR), hoping to reduce vole populations and damages and increase profits despite the investment required \citep{Stenseth2003}. They do so according to protocols, equivalent to farmer functional responses (hereafter FFR), that involve varying amounts of AR spread in response to different thresholds in vole densities. 

Voles and many small rodents are perceived as pests, but they are also keystone species, crucial to the functioning of grassland ecosystems, as well as being the prey of numerous predators, including species of conservation concern \citep{Delibes-Mateos2011, Coeurdassier2014}. Their population cycles create pulses of resources crucial to the viability of a wide range of resident predators and the aggregation of mobile avian vole predators \citep{Korpimaki1991}. The smallest mustelids (e.g. weasels Mustela nivalis) are specialist vole predators. Their numerical response has been shown theoretically to be necessary for generating predator-prey cycles \citep{Hanski1991}. They are said to be responsible for driving 3-to-5-year vole cycles in Fennoscandia (specialist predation hypothesis) \citet{Hanski1991}. Generalist resident predators like foxes (Vulpes vulpes) are expected to have regulatory and limiting effects on voles, owing to dietary plasticity that slows down vole population increase at low density \citet{Hanski1991}. Foxes do not show numerical responses to vole abundance \citep{Weber2002} but they influence the food chain through occasional killing and consumption of bite-sized mustelids. Mustelids form a small proportion of fox diet (0-10\%) but their offtake could represent a significant portion of the population (reviewed in \citet{Lambin2017}).

Anticoagulant rodenticides are non-selective toxicants with deleterious effects on non-target fauna (e.g. \cite{Coeurdassier2014}. Despite AR being exclusively licensed for rodent control, a large number of predator species are secondarily exposed to AR \citep{Sanchez-Barbudo2012}. Repeated consumption of dead and sub-lethally intoxicated voles reduced fox abundance in farmland in eastern France \citep{Jacquot2013} and ARs caused short-term declines in stoats in New Zealand \citep{Alterio1996}. Rodent-eating mustelid populations are affected by ARs given the pervasive levels of contamination reported \citep{McDonald1998}. Thus, there is little doubt ARs use inadvertently depresses predator populations. As they likely limit vole populations, it is essential to understand when AR use becomes counterproductive by altering the pest population dynamics, producing more frequent outbreaks and high residual vole abundance.

With the aim of understanding the potentially complex interactions between prey that are perceived as pest predators and farmers spreading rodenticide in response to vole abundance and their functional responses, we studied a simplified system inspired by cyclically fluctuating montane water voles (\textit{Arvicola scherman}), small mustelids (stoats, weasels) that mostly eat voles (specialists), and foxes (generalists), with voles and mustelids as food items. We used a process-based model using differential equations to explore 5 FFR types of AR spread, combining population dynamics, predator-prey interactions and rodenticide transfer across the trophic chains. Model parameters and FFR were inspired by farming systems in the Jura Mountains, Franche-Comté (France), the region of Comté cheese production. In Franche-Comté, farmers shifted from polyculture to almost exclusively grass production for milk used to produce cheese from the early 1970s \citep{Giraudoux1997}. Due to recurrent vole outbreaks and damages to grasslands, massive rodenticide treatments were implemented from the early 80s with consequences on non-target wildlife. Practices developed technically under pressure from public opinion, farmer unions and farmer technical organizations collaborating with researchers to find treatment regimes with less harmful consequences for biodiversity \citep{Delattre2009}. 

Hence, our main objective was to explore the properties of FFR in relation to varying vole density, to vole outbreak frequencies, to the guild of interacting predators and the effects on the global tri-trophic population dynamics. Additionally, we sought to establish under what regime of use the ARs deliver benefits to farmers without imposing damages to the ecosystem. 

\section{Methods}

We specified and parameterized the tri-trophic system of voles-mustelids-foxes with use of AR by farmers in response to vole density. We considered different FFR to assess AR upward transfer through the trophic web, considering the coupled dynamics of pulsed of AR spread and population dynamics. In all cases, parameterization units are on hectare-1 and day-1.

\subsection{Model for the tri-trophic dynamic}

We considered a tri-trophic system described by Fig. 1 and equations (1-5); parameterization is provided in Supporting Information 1. Voles, denoted V, were the primary prey, Mustelids, M, intermediate predators, and Foxes, F, were top predators, consuming voles and mustelids. For each species, the instantaneous variation of population size over time is:

\begin{equation}
\begin{array}{l l}
\dfrac{dV}{dt} & = V r_V \left( 1- \dfrac{V}{K_V}\right) - \Phi_{V,M}(V)M - \Phi_{V,F}(V,M)F\\[.3cm]
%
\dfrac{dM}{dt} & = \varepsilon_M \Phi_{V,M}(V)M - m_M M - \Phi_{M,F}(V,M)F\\[.3cm]
%
\dfrac{dF}{dt} & = F r_F \left( 1- \dfrac{F}{K_F}\right)
\end{array}
\end{equation}

The vole population followed a logistic growth rate, with rV  the maximal reproduction rate, fixed at $r_V = \ln(2 \times 600)/365$ per day, since montane water vole populations can increase from 0  to  600 individuals ha-1 or more \citep{Giraudoux1997} resulting in the equilibrium density being fixed at $K_V = 600$ individuals. 
%
The vole population was preyed upon by mustelid and fox populations. The vole consumption rate at different vole densities was described by functional responses ($\Phi_{V,M}$ for mustelids, $\Phi_{V,F}$  for foxes), see equations \eqref{eq:functionResponse}.
%
We treated small mustelids as vole specialist predators \citep{King2006}, assuming a Holling Type 2 functional response with attack rate $a_M$ and handling time $h_M$ (equation \eqref{eq:functionResponse}). We then represented foxes feeding on voles and mustelids by a multi-species functional response derived from Holling Type 3, referring to generalist feeding behaviour \citep{Baudrot2016}. For that, we denoted $a_{VF}$ and $a_{MF}$ the fox attack rate on voles and mustelids respectively. The parameter hF was the handling time for foxes.


\begin{equation}
\begin{array}{l l}
 \Phi_{V,M}(V)M = \dfrac{a_M V}{1 + h_M a_M V} \\[.3cm]
 %
 \Phi_{V,F} = \dfrac{a_{VF} V}{a_{VF}V + a_{MF}F} \times \dfrac{(a_{VF}V + a_{MF} M)^2}{1 + h_F(a_{VF}V + a_{MF} M)^2}\\[.3cm]
 %
  \Phi_{M,F}(V,M)F = \dfrac{a_{MF}V}{a_{VF}V + a_{MF}F} \times \dfrac{(a_{VF}V + a_{MF}M)^2}{1+h_F(a_{VF}V + a_{MF}M)^2}
\end{array}
eq:functionResponse
\end{equation}

Parameterization of functional responses was estimated to fit the daily satiation level of predators for handling times, and the observed 5-6 year vole cycles for attack rates (Supporting Information 1). We assumed foxes spent longer searching for voles than mustelids, based on each species diet and daily number of individuals captured. Therefore, $a_{VF}$ was considered larger than $a_{MF}$ and selected to produce 6-year vole cycles without AR.


Since we assumed small mustelids behave as specialist predators, we considered a numerical response linearly dependent on the functional response with parameter $\varepsilon_M$ (dimensionless) as conversion efficiency of prey into newborn predator (Supporting Information 2). The mustelid background mortality rate (i.e. all other reasons of death: ageing, disease, etc.) was $m_M$ (Supporting Information 2). 
We assumed foxes had a logistic growth rate function, parameterized with maximal growth rate $r_F = ln(3)/365$ and equilibrium density $K_F = 0.03$ [individuals.ha$^{-1}$] \citep{Ruette2003} without numerical response \citep{Weber2002}. 

\subsection{Model with rodenticide}

Fig. 1 represents the whole study system. Rodenticide is spread in grasslands during treatments, denoted $T_{Broma}(V)$, triggered by vole density $V$ (FFR described in Table 1).
%
Firstly, baits (50 mg.kg$^{-1}$  of bromadiolone, hereafter AR) were spread in grasslands at quantity 7.5 to 20 kg.ha$^{-1}$.
%
Such quantity, $C$, was available for voles, and a proportion disappeared in the environment at rate $k_0$ (set at $k_0 = 0.0815$) \citep{Sage2008}. The proportion consumed per vole, with rate $\kappa(C)$, was assumed to be an increasing function. The function $\kappa(C)$ was characterized by a maximum ingestion $M_{in}$, and a half-saturation constant for ingestion $D_{in}$ in [mg.kg$^{-1}$]:

\begin{equation}
\kappa_C= \dfrac{M_{in}\times C}{D_{in} + C}\dfrac{M_{in} \times C}{D_{in} + C} 
\end{equation}

For the toxicokinetics of AR (i.e. internal compound dynamics) leading to AR concentrations in animal body (voles, mustelids and foxes), we considered an uptake without biotransformation and time-regulated distribution, (i.e. AR concentration in the body of animals was instantly homogeneous) and that the whole body was consumed or scavenged without selection/rejection of tissues-organs. We also assumed disappearance including excretion of the parent compound and metabolisation, and that metabolites were non-toxic and/or excreted in the scats. 
For the toxicokinetics of AR ingested by voles, a fraction $C_V$ was assumed to remain active, stored mainly in vole livers and available to predators ingesting voles. The absorption rate of ARs ($\eta$) exceeds 50\% in less than 24h \citep{Jacquot2013}. The excretion rate from voles, $k_{out}$, $V$ was 0.4 day$^{-1}$ \citep{Sage2008}. The mortality rate through ARs was $\mu(C_V)$. Death through poisoning created a dead vole population ($V_d$) with AR concentration $C_V$. Dead voles could either be scavenged by mustelids/foxes or decompose at rate $d$. We assumed AR in dead voles disappeared from the system when voles decomposed. 
Mustelids could feed on live voles $V$, or non-decomposed dead voles Vd and we assumed a Type 2 functional response adapted for a multi-species functional response \citep{Baudrot2016}. Mustelids ingested AR with absorption rate $\eta_M$ (ratio between biomasses of voles, $B_V$, and mustelids, $B_M$) and the total of ingested voles (alive $V$ and dead $V_d$) was defined by function $\Theta_M (V,V_d=$ (see Supporting Information 2), in equation (4) which provides the rate of AR ingested by a mustelid:

\begin{equation}
C_V \times \eta_M \times \Theta_M(V,V_d)
\end{equation}

A fraction of AR ingested was accumulated in weasels while the rest was excreted with rate $k_{out,M}$. AR ingestion induced weasel lethal poisoning at rate $\mu_M(C_M)$, additive to natural mortality rate $m_M$.
AR was ingested by foxes with a rate proportional to the functional response of foxes to voles, dead voles and mustelids. Foxes also accumulated AR available in their prey, resulting in upward AR transfer in the trophic chain. Foxes accumulated AR in concentration $C_F$. A fraction of AR was excreted by foxes at rate $k_{out,F}$ \citep{Sage2008}. AR caused fox mortality at rate $\mu_F$ ($C_F$) \citep{Sage2010}.
We used log-logistic equations for describing dose-dependent mortality of animals exposed to AR. Vole and predator mortality rates due to AR $\mu_X(C_X)$ ($X$ referring to the considered species) were expressed by equation (5):

\begin{equation}
\mu_X(C_X) = \dfrac{1}{\text{period of time}} \times \dfrac{1}{1+ (LD_{50}/C_X)^H}
\end{equation}


where LD50 was the daily median lethal dose (50\% of population dying) with different values for voles (2 mg kg$^{-1}$), mustelids (2.1 mg kg$^{-1}$) and foxes (0.5 mg kg$^{-1}$, Supporting Information 1). $H$ is the Hill's coefficient, modulating the curve steepness and was estimated to fit the sparse data we have (Supporting Information 1). Mortality rates considered the duration of AR toxicity from experiments of up to 6 days \citep{Sage2010}.

\subsection{The farmer functional responses explored through simulation}

We considered a range of realistic FFR spanning treatments during vole outbreaks only as a precautionary approach; in which treatments only takes place at intermediate or low vole density threshold. These scenarios are inspired by historic and contemporary protocols of bromadiolone use to control montane water voles in Franche-Comté, but also representative of practice globally (Table 1) \citep{Delattre2009}. To check the influence of foxes and intraguild predation on the system’s dynamics, we simulated scenarios with and without foxes.

Our simulations tracked the linked vole-mustelid-fox dynamics for 40 years, after a “burn-in” period of 10 years to reduce dependency of results upon initial conditions, to observe several vole cycles and to characterise AR effects on these species population dynamics. This burn-in period also had AR treatment triggered at specific vole densities and with a given rodenticide quantity for each FFR (see Table 1). The burn-in phase was selected according to a set of simulations with different initial conditions. Those simulations showed that in a given FFR (i.e., same threshold of vole density and amount of AR spread), the dynamics of the population were converging toward a similar pattern. 
In order to assess the impact of FFR on both agriculture and conservation interests, we estimated the following cost functions: (i) Number of treatment events per FFR ; (ii) Cumulative amount of AR (kg) ; (iii) Proportion of time when the AR-induced mortality of mustelids higher than 50\% (i.e. lethal exposure profile killing 50\% of mustelid population) ; (iv) Proportion of time when the mortality of mustelids was higher than 50\% due to natural mortality; (v) Proportion of time when the vole density was below 50 voles ha-1, as a proxy for time when forage grass grows with low herbivore influence; (vi) Mean vole, mustelid and fox densities.

\section{Results}

\subsection{Population dynamics}

Allowing for mortality by predators ingesting AR-poisoned voles changed the outcome of predator-prey dynamics involving vole, mustelid and fox populations. Secondary poisoning led to a rich spectrum of emergent dynamics according to the FFR to vole abundance.  
Without AR (FFR a), vole dynamics were regulated by mustelid predation that gave rise to 6-year cycles (Figs. 2 and 3; maximal vole and mustelid densities were around 600 ind.ha-1 and 2 ind.ha-1 respectively). The fox population remained at its carrying capacity (i.e., 0.03 ind. ha-1, see Supporting Information 1).  

Under FFR b (high vole density threshold triggering treatment and high AR amount per treatment) and FFR d (intermediate threshold and low AR amount), vole dynamics were sequentially regulated by either AR treatments, which we refer to as farmer-regulated phase (FR in Figures 2 \& 3), or by mustelids, mustelids-regulated phase (MR), alternately (MR in Figures 2 \& 3). Farmer-regulated periods started when densities of living voles triggered treatments. This produced sudden declines of live voles followed by increases of dead voles. However, the vole population re-grew quickly which triggered frequent further treatments and pulses of availability of contaminated (both live and dead) voles. In FFR d (intermediate threshold and low AR), vole declines were not as deep as when pulses of AR amount were high (FFR b), owing to the reduced rodenticide amount per treatment. Mustelids and foxes also experienced AR-induced declines during this period. Under FFRs b (high threshold density, high AR) and d (intermediate threshold, low AR), mustelid-regulated periods started when mustelid numbers grew slowly to a peak, which depressed vole density, precluding rodenticide treatments and releasing the fox population from secondary poisoning, such that its abundance rebounded. Vole depletion by mustelids and subsequent mustelid declines allowed the vole population to grow again up to threshold densities and initiated a new period of regulation by farmers.
Vole dynamics were permanently regulated by AR treatment under FFR c (intermediate threshold, high AR, Figs. 2c, 3c). Populations of live and dead voles experienced high frequency fluctuations (around 2 peaks every year) driven by AR. As AR treatments were frequent, being triggered by voles peaks, contaminated dead voles were always abundant (peaks at 90 ind.ha-1), and mustelid and fox densities remained low (mustelids, converging to 0 ind. ha-1; foxes, 0.003-0.006 ind. ha$^{-1}$) with only short-term fluctuations (2 peaks every year, following vole cycles).
With FFR e (low threshold, low AR), vole populations were maintained by farmers at around 50 voles ha-1 (Figs. 2e and 3e). The population of live voles was buffered by treatments. Additionally, whenever voles reached densities triggering treatment, predator populations experienced strong declines. However, fox densities (0.01-0.015 ind. ha-1) were higher compared to FFR c (intermediate threshold, high AR), reflecting the reduced amount of AR used (7.5 kg) and transferred to foxes as there were lower vole densities. The maximum numbers of dead voles under this FFR e was relatively low (highest around 15 ind.ha-1) but, due to frequent treatments, there was a steady replenishment of contaminated dead voles. This, in turn, induced mustelid and fox mortality and population declines (Supporting Information 3). Additionally, low availability of live voles triggered small mustelids mortality through starvation down to abundances similar to those resulting from AR use (Supporting Information 3). 

\subsection{Comparison between farmer functional responses in terms of cost functions}

As expected, spreading AR generally reduced vole densities (Table 1) but the extent to which it also affected predators and the costs and benefits for conservation and farmer’s interests varied widely. 
The FFR b (high threshold, high AR) implied the lowest number of treatments (mean 0.87 year-1), while FFR d (intermediate threshold, low AR) resulted in the lowest amount of AR used (mean 14.5 kg ha-1 year-1). The FFR e (low threshold, low AR) had the lowest mustelid mortality due to AR (no instance in 40 years) but maximised starvation-induced mortality of mustelids (Table 1). Additionally, FFR e (low threshold, low AR) had the highest proportion of time with <50 voles ha-1, (96 % of study period). When considering the effectiveness in reducing mean vole densities, FFR e (low threshold, low AR) ranked highest followed by FFR c (intermediate threshold, high AR). The FFR d (intermediate threshold, low AR) and b (high threshold, high AR) allowed the highest mean densities of mustelids and foxes respectively. 
Comparison of these protocols indicates that FFR d (intermediate threshold and low AR) delivered the best compromise between farm production and conservation interests (Table 1). It had the lowest AR quantity applied (mean 14.5 kg ha-1 year-1) and the highest mean densities of mustelids (0.47 ha-1) and foxes (0.02 ha-1). The number of treatments was among the lowest values (1.6 treatment year-1). Its drawback was that it only suppressed voles at low densities (<50 voles ha-1 for 21% of time).  Conversely, emergent mean vole densities were low to intermediate (122 voles ha-1). 
Other FFR regimes were poorer compromises either because mustelids were kept at low densities (FFR  c, intermediate threshold, high AR, and FFR e, low threshold, low AR) or had relatively high mean vole densities (229 voles ha-1, FFR b, high threshold, high AR). In summary, FFR d (intermediate threshold, low AR) was the best compromise because it minimized the impact over predator populations contributing to the natural reduction of vole densities with the lowest quantity of AR applied for farm production interests.

\subsection{Influence of intra-guild predation}

Fig. 4 shows the system dynamics under FFR d, where successive farmer-regulated and mustelids-regulated phases occured. This simulation shows that the removal of foxes did not eliminate the successions of mustelids-regulated and farmer-regulated phases. However, the mustelid regulated period allowed short-term peaks of voles, suggesting the emergence of a classical one-predator - one-prey cycles interrupted by a farmer-regulated period. Without foxes, population dynamics of mustelids presented a more chaotic behaviour, while it presented regular cyclic pattern with fox occurrence. Therefore, this simple model suggests a stabilizing role of a generalist predator (foxes in this system) during the mustelid-regulated period. At the end of the mustelid-regulated period, foxes strongly contributed to vole mortality and, to a lesser degree, mustelid mortality. Consequently, the removal of foxes implied less predation over voles during the mustelid regulated period, and short-term vole releases from mustelid predation. The 2-year rolling mean of vole density (blue lines in Fig. 4) illustrates the change of regime from farmer-regulated to mustelids-regulated phases. Indeed, the amplitude of averaged vole densities (i.e., the amplitude of vole cycles for the 2-year rolling mean) was relatively stable at the beginning of farmer regulation period and then suddenly decreased to become minimal before sharply increasing, announcing the regime change. These changes in density amplitude may be used as an early-warning signal of the regime transition. 

\section{Discussion} 
Considering that rodenticide kills not only voles but also their predators through secondary poisoning, our models show that AR profoundly changes the outcome of predator-prey dynamics involving vole, mustelid and fox populations beyond what mere intuition could elucidate. Our study reveals how the dual influences of the amount of pesticide spread and the vole density threshold triggering AR spread drives (i) pesticide spreading frequency, (ii) predation ecosystem service, and subsequently (iii) the control of pest outbreaks. Two types of a rich spectrum of emergent dynamics, including farmer or mustelid regulation deviating from classical predator prey dynamics arose because poisoned voles acted as “super-predators”. 

\subsection{Modelling farmer regulation into a classical predator-prey system}
 
The threshold functional response of farmers deciding when to apply varying amounts of rodenticides according to prevailing vole density was crucial in selecting the emergent ecosystem dynamics, resulting in much variability in ecosystem and conservation and farming production interests. In the ecosystem our models depict, farmers spreading rodenticide not only depleted vole prey exploited by specialist and generalist predators but also created pulses of lethally or sub-lethally poisoned voles that became super-predators by poisoning their predators.  Arguably this set of ecological interactions has similarities with circumstances where a pathogen affecting prey species also infects predators, as in the case with the flea vectored plague (\textit{Yersinia pestis}) infecting prairie dogs (\textit{Cynomys} spp.) and black footed ferrets (Mustela nigripes) in central US \citep{Matchett2010}. However, to our knowledge, the behaviour of such tri-trophic model with multiple reciprocal interactions has not been explored. This is despite the obvious relevance to the management of the globally widespread circumstances where keystone small mammals are poisoned and may secondarily poison their predators \citep{Delibes-Mateos2011}.  
Under the “reference” scenario without AR spreading (FFR a), we assumed a predator-prey cycle which is a plausible pattern thoroughly explored theoretically \citep{Hanski2001}. There is no controversy on the role of small mustelids tracking vole dynamics, though it is not yet well understood whether predation may drive steep declines \citep{King2006}. Parameters of the reference scenario for our predator-prey model were realistic and tuned to generate population fluctuations similar to those observed in the studied cyclic system \citep{Delattre2009}.  The addition of pulses of rodenticide and their toxicokinetics in vole and predators are based on previous experiments with bromadiolone, a widely used AR, ensuring biologically realistic functional forms and their parameterization. Irrespective of the FFR considered, the frequency of vole cycles dramatically increased compared to the reference scenario, except during mustelid-regulated phases emerging under some FFR scenarios.

\subsection{How specialist predators may protect generalists from poisoning}

An interesting model behaviour was seen with FFR b (high vole density threshold and high AR) and FFR d (intermediate vole density threshold and low AR) with farmer- and mustelid-regulated phases alternating with low frequency. Such flipping between alternative states in population dynamics has been previously described in predator-prey model where weasels rely on a primary prey and entrain the dynamics of secondary prey \citep{Hanski1996} but not for the kind of indirect interaction we explore here. It further demonstrates that adding biological realistic complexity to simple models may drastically change the emergent properties of trophic interactions.
From these scenarios, we understand that the emergence of successive farmer and mustelid phases is neither driven by vole density threshold alone nor by AR amount, but instead by a subtle combination of both. The modelling description of these patterns uncovered the dual key roles of mustelids on fox dynamics, as intraguild competitors and as a vector for poisoning. This led to a surprising form of facilitation for foxes: mustelids protect foxes from collapses. The establishment of such a response can be described in 3 steps. Firstly, low mustelid densities inhibit their regulation of voles and contribute to farmer AR use. In line with empirical evidence, the latter directly impacts foxes by poisoning \citep{Jacquot2013}. Secondly, fox predation on mustelids is reduced, and with an intermediate AR amount, this allows mustelids to slowly recover. Vole outbreaks and subsequently farmer treatments are then gradually delayed, benefitting mustelids recovery. This is the point of transition from farmer to mustelids regulation, starting the third step: mustelids increase faster, suppressing vole densities and precluding the need for AR treatments, and eventually indirectly allowing fox population growth. Our finding that complexities in trophic interaction, induced by the poisoning of predator by poisoned prey, may cause the system to flip between alternative states is novel and robust. However, given we only explored deterministic versions of our models, any inference on the frequency of flipping between states should be cautious given the inherent stochastic nature of natural and farmland environments. If such dynamics occur within real farming systems, flipping between states is unlikely to emerge with regularity (~40 years) where many other factors impact population dynamics. Moreover, under our heuristic modelling framework, we have shown that the removal of fox population induced chaotic dynamics of voles and mustelids (Fig. 4). While generalists are known to have stabilising effect \citep{Hanski1991}, the benefit of specialist predators imparted to generalist predator and resulting increase in the prevalence of intraguild predation would be difficult to detect in empirical studies. Nevertheless, other generalist predators such as the endangered red kite (\textit{Milvus milvus}) which feed on voles opportunistically, occupy areas with bromadiolone treatments and are also affected by rodenticides \citep{Coeurdassier2014} and may therefore also benefit from the presence of mustelids in the ecosystem.

\subsection{How region-wide vole extirpation may inhibit ecosystem services}

Under FFR c (intermediate vole density threshold and high AR) and FFR e (low vole density threshold and low AR), the whole system was solely driven by farmer regulation, whereby the chronic use of AR completely suppressed the pest-regulation ecosystem service of predators. It has previously been shown empirically that repeated rodenticide treatments are highly detrimental to the populations of predators and reduce their densities  
\citep{Jacquot2013}. Secondary poisoning of predators is an established reality \citep{Berny2007}.  Through modelling, we formalised the insight that some poison deployment protocols, including those presently used in the empirical system which motivated our study, are counterproductive if employed on a large scale, suppressing natural predator regulation of pest rodents. It has been long known that poisoning rodents with AR permeates the food chain at peak abundance and achieves little in terms of protecting crops and may have strong deleterious impact (FFR b and c) \citep{Olea2009}. In Franche-Comté, a change in treatment protocols, from controlling voles at high densities to low-intermediate densities, has reduced the mortality of non-target species (including foxes) \citep{Jacquot2013}. Nevertheless, deployment regimes of pesticides that can contaminate the food chain should also enable periods of time which allow predator populations to rebound and avoid extirpation from the ecosystem. We have shown that, over time, farmers who strictly maintain voles at low density thresholds would suppress predation services provided by vole predators and instigate pesticide dependence. In addition, small mammals like voles certainly have ecosystem functions. Our results also suggest that the presence of small mustelids in ecosystems is beneficial for biodiversity conservation (see above) and agriculture interests. Given the importance of vole cycles and their trophic interactions, it is desirable to maintain vole population fluctuations of sufficient amplitude to maintain ecosystem processes.

\subsection{Managing rodents and ecosystems}


Presently, in Franche-Comté, farmers relying on bromadiolone alone can only treat pre-emptively when voles are at low densities (FFR e) whereas those using alternative methods without pesticide are allowed to spread AR in low quantity up to intermediate vole densities (FFR d). Spreading AR in low quantity seems superficially desirable, but our heuristic model, assuming an idealised homogenous landscape, show this is associated with frequent treatments. Consequently, it would induce a near permanent availability of a small number of intoxicated voles which, combined with low availability of non-contaminated voles, would reduce predator populations. Therefore, the extreme situation of using a low vole density threshold (FFR e) at a large scale is undesirable because it depletes the prey resources of foxes and mustelids and their populations. Triggering treatment at intermediate vole density with a low amount of AR (FFR d) allowed for temporal refuges, i.e. longer periods free of rodenticide necessary for predator densities to rebound while simultaneously avoiding episodes with high vole density, as required by farm production interests. Under a landscape management approach, such temporal refuges could be spatial refuges, with parts of the landscape free of pesticides where predator populations can recover.

Our key result and the basis for management prescriptions is that allowing for refuges where voles are not poisoned and allowed to persist at medium-high densities such that they can be exploited by mustelids is crucial for predator population recovery and preserving the ecosystem services mustelids deliver.  Treatment regimes allowing refuges seems compatible with both conservation and farming interests. A critical insight is to avoid potential side effects of chronic low-dose AR prescription (e.g., depletion of community services, stimulation of resistances), as is well known with antibiotics, by demanding regularly long-term period without treatment. However, combining chronic treatments and long periods free of AR may be difficult to achieve in real systems. Our model only considers temporal refuges, and the conceptualization of untreated areas as equivalent to triggering treatment at intermediate vole density cannot provide guidance on the size of these spatial refuges. Nevertheless, while management of voles is implemented at the scale of fields, mustelids and foxes roams over much larger areas  \citep{King2006}, such that large refuges with medium-high vole densities voles would be required while allowing to maintain low-voles density at very local scale.

\subsection{Conclusion and perspectives}

Our process-based model revealed pesticides that permeate the food chain upward can lead to diverse population dynamics with alternative states regulated by predators and farmers. It also shows that the practice currently promoted to use low-dose AR treatments at low vole density could be limited by the undesirable side-effects of leading to chronic application of AR on a large scale and the depletion of the vole predator community. This emerging question would benefit from a landscape modelling approach to characterize spatial refuges. We have also uncovered a counterintuitive mechanism whereby, owing to intraguild predation, mustelids could rescue foxes from poisoning. This suggest that contemporary Environmental Risk Assessment of pesticides that mostly consider one-species - one-compound experiments fail to capture the impact of pesticides on trophic links.  Assessing risk at the ecosystem level is empirically challenging such that process-based modelling can play a critical role. 

\section*{Authors’ contributions}
	
XL conceived the initial idea; all authors developed the concept; VB and JF developed the models and led manuscript writing; VB implemented the model and ran simulations. GC contributed treatment protocols; JF, VB and MC explored model parameters; XL, PG and MC contributed critically to drafts; all authors gave final approval for publication.


\section*{Acknowledgments}

JF benefited from a Marie Skłodowska-Curie fellowship (European Commission, project "VOLES", 660718). VB was employed with this project funds. We are very grateful to Deon Roos for reviewing drafts. We thank Alessandro Massolo, Thibault Moulin and Francis Raoul for helpful suggestions. This work benefited from long-term data collected at Zone atelier (ILTER) Arc jurassien (http://zaaj.univ-fcomte.fr) and its financial support.



\section*{References}

%\bibliographystyle{elsarticle-harv}
\bibliographystyle{plainnat}
%\bibliographystyle{unsrt}
\bibliography{trophToxNTO.bib}


\end{document}
